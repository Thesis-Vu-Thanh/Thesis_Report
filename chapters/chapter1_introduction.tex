\chapter{Introduction}
\label{chap:introduction}
	\textit{In this chapter, we introduce issues in the field of cyber security, clarifies the necessity of building a firewall system. Next, we present an overview of the goals, challenges and structure of the thesis.}
\minitoc

\section{Motivation and problem statement}
\label{sec:motivation}
Websites have become necessary for every business, brand, institution, organization, and individual. Web-based applications offer the general public and enterprises fast and simple services, they may be used for social media, email, banking, online shopping, education, or entertainment. Web applications are popular for a variety of reasons\footnote{Amy Bros. Sysprobs. \textit{Why are Web Applications Becoming Popular?}. July 2022. \url{https://www.sysprobs.com/web-applications-becoming-popular}}. For starters, web-based applications are convenient for users because they can be accessed from any location that has an internet connection. Second, a web application is a low-cost method for corporations because it does not require any special hardware or software and can be easily scaled up or down as needed. Furthermore, web-based apps frequently offer subscription-based models, which can be more cost-effective for businesses in the long run. Third, because web apps do not need to be downloaded or installed on a device, they are generally faster and more responsive than native apps. These are instead run on remote servers and accessed through a web browser. As a result, web apps can fully utilize the most recent advances in server-side technology, which helps in performance and speed. Why would someone want to attack web applications? Web applications are an attractive target for remote attackers because they can access the web app anywhere with an internet connection. Besides, they frequently handle sensitive information, such as login credentials and financial data. In short, web applications are targeted because they are visible, accessible, and offer a multitude of potential payoffs for an attacker. Cybercriminals can attack web applications for many reasons, including system flaws caused by incorrect coding, misconfigured web servers, application design flaws, or failure to validate forms. These flaws and vulnerabilities enable attackers to gain access to databases containing sensitive information. 


The action conducted by attackers, which can do harm to the web application can be called cyber attacks. Cyber attacks are malicious attempts to access a person's or an organization's computer systems, networks, or data without authorization. The vulnerability to cyberattacks and our reliance on technology and connectivity are growing in tandem. Unlike viruses that would shut down a system for a few hours a few years ago, the consequences of cyber attacks today can include stolen data, destroyed networks, and thousands, if not millions, of dollars in recovery efforts. Cyber attacks can harm businesses, governments, and society\footnote{CEI-The Digital Office. \textit{The Consequences Of Cyber Attacks And Their Impact On Cybersecurity}. \url{https://www.copycei.com/consequences-of-cyber-attacks}}. Among them, businesses and companies are more prone to being attacked by cybercriminals because they have more holes and gaps in their security that make them vulnerable to attacks. Cyber attacks affect a company's productivity, revenue, and reputation. Regarding productivity, nearly every business that suffers a cyberattack must suspend part or all of its operations until the attack is resolved, whether by paying a ransom, removing the malware from the device, network, or system, or restoring a backed-up version of its system. In terms of revenue, the costs of a cyberattack can wreck the economy of a company. The average cost of a data breach for a small to medium-sized business is massive, whether it has to shut down operations for several days, pay a ransom, lose data, replace devices, or pay a security expert to clean all malware out of the system or network. The most serious effect of a cyberattack is the loss of reputation. The most critical consequence of a cyberattack on a business is a loss of reputation. Consider the recent data breaches at Equifax, Target, and J.P. Morgan Chase, which led to the loss of customer data such as social security numbers, account details, and credit card numbers. Despite having the resources to recover, most businesses can not recover from security breaches because they lose their clients' trust and thus business. The biggest concern of a cyberattack on a government entity is the enormous volume of stolen data. This data could include everything from military and national security information to private data about civilians, which could be sold on the dark web and misused by terrorist groups. When cyberattacks happen, they badly affect practically every part of society, whether it's a large corporation or a small business. Consumers pay the price when enterprises, companies, and even nonprofit service providers like hospitals are forced to cover the costs of a cyberattack. There will be shortages that the customer will have to suffer when a company is restricted from providing its service as a result of cyber attacks or data breaches.


Let's take a look at Parachute's statistics\footnote{Parachute. \textit{Cyber Attack Statistics to Know in 2023}. \url{https://parachute.cloud/cyber-attack-statistics-data-and-trends/}}. Healthcare, throughout the past 12 years, this sector has experienced the most costly data breaches, the costs have even increased by 41.6\% from 2020 until 2022. At least 849 healthcare cybersecurity incidents and 571 data breaches were reported in 2022. The average financial loss due to data breaches in healthcare has skyrocketed from around USD 9 million to USD 10.10 million (2022). In the Finance industry, phishing attacks against banks and other financial institutions held the largest share, accounting for 23.2\% of all cyberattacks targeting the financial sector. In the first quarter of 2022, ransomware assaults increased by 35\% in the financial sectors. On average, financial organizations bore the second-highest data breach costs, at USD 5.97 million, just behind healthcare institutions (2022). 

From the above situation, we can see that tools like Firewalls play a crucial role in organizations' security systems. They are a powerful tool for preventing malicious traffic from entering or leaving an organization’s systems. Our group wanted to make some efforts to strengthen firewalls' cyber security, so we decided to choose this topic - Building a Security module for Web firewalls.
\newpage
\section{Objectives}
\label{sec:objectives}
Given the necessity of cybersecurity in this era, we chose to construct a firewall add-on. This thesis will concentrate on identifying malicious requests that pass through the firewall. We want to apply machine learning to assist the WAF with processing a tremendous amount of requests per second swiftly and efficiently. The developed module should be able to validate the request and identify whether or not it is an attack.

To eliminate the high FP, an inherent weakness of rule-based WAF, and assure the speed of the WAF, which must react nearly instantly to deliver a consistent customer experience, the module will use two fast and simple machine learning models. Each model will run independently to generate an output, then combined the result to achieve an accuracy of at least 95\% as well as low latency of 5 milliseconds\footnote{Abdalslam. \textit{Web Application Firewalls (WAF) Statistics, Trends And Facts 2023}. 
\url{https://abdalslam.com/web-application-firewalls-waf-statistics}}for the WAF.

\section{Challenges}
\label{sec:challenges}
In this topic, we have encountered some problems such as:
\section{Tentative structure of the thesis}
\label{sec:structure}
	\newcommand\nextintro{\\[4mm]}
	The content of the thesis proposal is demonstrated by these 8 following parts: \nextintro
	\hyperref[chap:introduction]{\textbf{Chapter 1}} introduces issues in the field of cyber security, and clarifies the necessity of building a firewall system. Next, we present an overview of the goals, challenges, and structure of the thesis.\nextintro
	\hyperref[chap:backgrounds]{\textbf{Chapter 2}} introduces the background knowledge of this thesis, including information about Web Application Firewalls, Machine Learning Models, and Deep Neural Networks.\nextintro
	\hyperref[chap:literaturereview]{\textbf{Chapter 3}} mentions some related studies on the use of machine techniques to detect malicious requests, their strengths and weaknesses, and evaluate an appropriate approach for this thesis.\nextintro
	\hyperref[chap:dataset]{\textbf{Chapter 4}} provides the dataset that would be used to train and evaluate the malicious request validator, its problems, and pre-processing details\nextintro
	\hyperref[chap:proposed_approach]{\textbf{Chapter 5}} displays the problems as well as the malicious request validator's input and output. Then we offer the design and architecture of the problem's solutions.\nextintro
	\hyperref[chap:implementation]{\textbf{Chapter 6}} shows the implementation and deployment processes, including system frameworks and system specifications.\nextintro
	\newpage
	\hyperref[chap:experiments]{\textbf{Chapter 7}} discusses the division of the data set before training, assessment techniques, and experimental results. Compare the outcomes of the experiments we suggest to the reference experiments from relevant works.\nextintro
	\hyperref[chap:conclusion]{\textbf{Chapter 8}} summarizes the results for the thesis until now. Finally, we want to present a working plan to improve the thesis.\nextintro
	% \ref{appendix:thong_so_tap_du_lieu_3d_ircadb_01} Trong phụ lục này, chúng tôi trình bày bảng thông số chi tiết về tập dữ liệu \textit{3D-IRCADb-01}. Bảng thông số này được lấy từ trang chủ của Viện nghiên cứu chống ung thư đường tiêu hoá \textit{IRCAD}\nomenclature{IRCAD}{Research Institute against Digestive Cancer}. \nextintro
	% \ref{appendix:ke_hoach_thuc_hien_luan_van} Trong phụ lục này, chúng tôi trình bày chi tiết kế hoạch thực hiện đề tài bao gồm các công việc đã thực hiện trong giai đoạn Đề cương luận văn và giai đoạn Luận văn. Xây dựng kế hoạch cụ thể là cơ sở quan trọng giúp chúng tôi hoàn thành tốt luận văn này.