\chapter{Literature review}
\label{chap:literaturereview}
	\textit{This chapter mentions some related studies on the use of machine techniques to detect malicious requests, their strengths and weaknesses, and evaluate an appropriate approach for this thesis.}
\newpage	

Malicious request detection has been a focus of recent studies, with several malicious request detection systems developed. They propose solutions for particular or different types of harmful requests. For example, reference \cite{Joshi_sqli} show us their solution to SQLI attack which is one of the most common malicious requests. Joshi and Geetha \cite{Joshi_sqli} detect SQLI attacks using the Naive Bayes machine learning algorithm. Or \cite{sqlmap} automatically generates malicious requests containing SQL injection to exploit flaws and take over of database servers. Reference \cite{sqli_ML} train a Support Vector Machine(SVM) model to detect SQLI attacks.\\ \newline
Fawaz Mereani and Jacob Howe et al\cite{Mereani2018} have demonstrated that SVM, k-NN, and Random Forest can be used to build classifiers for XSS coded in JavaScript giving high accuracy (up to 99.75\%) and precision (up to 99.88\%) when applied to a large real-world dataset. One especially intriguing component of this study is that, unlike other studies, a binary measure was employed for all features. This has resulted in greater accuracy and precision than previous tests employing weighted measurements.\\ \newline
Laughter et al \cite{laughter2021} integrated the HTTP request features in the process of visiting the website into the detection feature set. By extracting the content of each field in the request header and request body, classification methods such as decision trees and SVM were used to complete the research.\\
\newline
Alshammari, Amirah and Aldribi, Abdulaziz et al \cite{Alshammari2021} present a reliable model running in Real-time to detect malicious data flow traffic on the cloud depending on the ML supervised techniques based on the ISOT-CID dataset that contains network traffic data features. When tested using cross-validation and split-validation, DTREE and Random Forest both produced the best accuracy results. Their two models did not fail in any of the classification processes used to assess different portions or folds of the dataset.\\
\newline
This group of researchers \cite{s22093373} proposed an approach to detect malicious URLs using ensemble learning. They use TF-IDF to pre-process input URLs, then applied Rain Forest Ensemble-Based for prediction and an artificial neural network (ANN) classifier was constructed for decision making. Results show that this approach significantly improved the detection performance, achieving 96.80\% compared with the best 90.4\% achieved by the URL-based features. The false-positive rate was significantly decreased to 3.1\% compared with 12\% performed by the URL-based model. \\ \newline
Khoi. Le et al \cite{Khoi} suggested an approach of extracting WAF rules and trained a machine learning with the decision model independent from the rules themselves. This makes the model more self-reliant and the overall result more neutral. Our module is inspired by this approach.\\ 
