\chapter*{Preface}
\thispagestyle{fancy}
\addcontentsline{toc}{chapter}{Preface}
\label{tom_tat}
\hspace*{5cm}

With the increased exchange of information and other activity on the World Wide Web, the Web has become the primary platform for attackers to cause havoc. Effective methods for detecting Web threats are crucial for ensuring Web security. With the explosion of data, it is becoming more challenging to detect and prevent cyber threats; however, the utilization of machine learning-based ways to fight cybersecurity is on the rise, with broad uses of machine learning in cybersecurity providers. 

In this thesis, we propose a machine learning-based strategy to cyber-attack defense. Typically, rule-based WAFs have been widely employed. They do, however, have a significant false-positive rate. Therefore, we are building a machine learning-based module to validate requests which have been labeled suspicious by WAFs to enhance the WAFs and provide better surveillance. 

The malicious request validator is based on the assumption that genuine requests to a website generally fit into the same category. The module uses a Convolutional Neural Network to classify the suspected request and evaluates whether it falls into the same category as the standard requests observed. The ultimate decision on whether to block a request is made using this result and combining it with the request content analyzer.

% The dataset for phishing detection is collected from Polaris Infosec, a Web Application \& API Protection (WAAP).

The false positive rate (or the false alarm rate) is the most crucial requirement for modules in experimentation. Precision and accuracy are other key criteria. We also accounted for processing time with the suspicious request detector, as WAFs are supposed to be instant.
	
\cleardoublepage