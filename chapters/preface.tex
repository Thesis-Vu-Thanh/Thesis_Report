\chapter*{Preface}
\thispagestyle{fancy}
\addcontentsline{toc}{chapter}{Preface}
\label{tom_tat}
\hspace*{5cm}

With the increased exchange of information and other activity on the World Wide Web, the Web has become the primary platform for attackers to cause havoc. Effective methods for detecting Web threats are crucial for ensuring Web security. With the explosion of data, it is becoming more challenging to manually detect and prevent cyber threats; however, the use of machine learning-based ways to fight cybersecurity is on the rise, with broad uses of machine learning in cybersecurity providers. 

In this thesis, we investigated and evaluated the approaches of machine learning in WAF, evaluating the benefits and drawbacks of each related work in recent times. Then we proposed a machine learning-based WAF to defend web applications from cyber attacks. Typically, rule-based WAFs have been widely employed. They do, however, have a significant false-positive rate. As a consequence, we are building a machine learning-based WAF to validate requests that have been labeled suspicious by WAFs to enhance the WAFs and provide better surveillance. 

The malicious request validator is based on the assumption that genuine requests to a website generally fit into the same category. The module uses a Convolutional Neural Network to classify the suspected request and evaluates whether it falls into the same category as the standard requests observed. The ultimate decision on whether to block a request is made using this result and combining it with the Logistic Regression request content analyzer. 

The false positive rate (or the false alarm rate) is the most crucial requirement for modules in experimentation. Precision and accuracy are other key criteria. We also accounted for processing time with the suspicious request detector, as WAFs are supposed to be instant. The best experimental results for our machine learning-based WAF tested on the CSIC 2010 request dataset is almost a zero false positive rate with an accuracy of 85.23\%, compared to only 66.98\% when we combined the machine learning models with rule-based WAF, in this case, ModSecurity. However, our WAF is still not competitive enough with other studies in the same WAF approach, nor does it meet the response time criteria. And it is clear from the aforementioned experiments that the machine learning-based WAF approach has certain advantages over the other techniques and can be applied adaptable in a variety of circumstances.
	
\cleardoublepage