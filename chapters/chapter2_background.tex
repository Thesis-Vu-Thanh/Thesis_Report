\chapter{Backgrounds}
\label{chap:backgrounds}
	\textit{This chapter introduces the background knowledge of this thesis, including information about Web Application Firewalls, Machine Learning Models and Deep Neural Networks}
\minitoc

\section{Web Application Firewall} 
\label{sec:waf}
	
\subsection{What is WAF ?}
\label{subsec:waf_def}
WAF stands for \textbf{Web Application Firewall}. This firewall solution commonly monitors data packets and filters them for the presence of malware or viruses. It performs the data monitoring/filtering for to and from data packets.  

The WAF tool can be distributed using network-based, cloud-based, or host-based architectures. It needs a reverse proxy to make sure that one or more web apps are in front of it while facing forward. 

It can be utilized either alone or in conjunction with other applications. WAF may function at a lower level or a higher level depending on the requirement\footnote{Wallarm. \textit{WAF Meaning}. \url{https://www.wallarm.com/what/waf-meaning}}.


\subsection{How does WAF work?}
\label{subsec:waf_work}
As previously stated, WAF is deployed at the application layer and serves as a two-way firewall. At work, WAF monitors HTTP or HTTPS traffic entering or exiting a specific web app. When WAF detects a malicious object in the traffic, it activates and destroys it. \\
\begin{figure}[!h]
   
	\centering
	\includegraphics[width=\linewidth, height=10cm,keepaspectratio]{figures/WAF.JPG}
	\caption{Normal users (top-left and bottom-left) are permitted access to the server with a WAF enabled, but attackers (middle-left) are prevented from doing so.}\label{Fig:Data1}
  
\end{figure}
\\
WAF predefined what is malicious and what is not to make the process easier. WAF adheres to these guidelines throughout the process. WAF primarily analyzes the GET and POST portions of HTTP traffic. GET retrieves data from the server, whereas POST directs data to the server to change its original state\footnote{Wallarm. \textit{WAF Meaning}. \url{https://www.wallarm.com/what/waf-meaning}}.

\subsection{WAF vs Firewall}
\label{subsec:versus}
On the surface, the firewall and WAF have seemed to be the same. They also share several characteristics. They are not, however, the same. There are some key differences. Understanding the distinctions is critical when attempting to put anyone into action. A firewall is a catch-all term for a variety of firewalls used to protect computer networks. While in use, it filters data packets. Firewalls can be distinguished from one another based on the level of security they provide and the method by which they are delivered. Only a few firewalls, for example, use packet filtering, whereas others use proxies, NGFW, or stateful inspection. WAF is better compared to proxy firewalls. There is, however, a distinction. The system is not protected from assaults that occur on a fundamental level by a firewall. The WAF concentrates on Layer 7 logic.

Web application firewalls, or WAFs, are primarily used to protect web applications from online threats. Work on it starts at the application layer. Therefore, WAF implementation is always done at the application layer, regardless of the strategy\footnote{Wallarm. \textit{WAF Meaning}. \url{https://www.wallarm.com/what/waf-meaning}}. 

	\begin{figure}[h!]
		\centering
		\includegraphics[width=\linewidth, height=10cm,keepaspectratio]{figures/waf3.png}
		\caption{Comparison between WAF and network firewall}
	\end{figure}

% \subsection{Receptive field} 
% \label{subsec:receptive_field}
% 	Receptive field\index{Receptive field} trong lĩnh vực thị giác máy tính được định nghĩa là vùng trên không gian đầu vào tương ứng cho một đặc trưng được trích xuất ở kết quả đầu ra. Receptive field của một đặc trưng được mô tả bằng điểm chính giữa và kích thước của nó.  Trong một receptive field, điểm ảnh càng gần tâm càng đóng góp nhiều vào đặc trưng đầu ra. Điều đó có nghĩa rằng, một đặc trưng không chỉ nhìn vào một vùng riêng biệt trong ảnh đầu vào mà còn tập trung nhiều vào vùng giữa của receptive field đó. 
	
% 	\autoref{fig:receptive_field} minh họa receptive field của toán tử convolution có kích thước filter\index{Filter} $3\times3$, stride\index{Stride} là 2 và padding\index{Padding} ``same''. Áp dụng toán tử này trên ảnh đầu vào có kích thước 5$\times$5 cho kết quả đầu ra có kích thước 3$\times$3 (màu xanh). Tiếp tục áp dụng toán tử này trên đầu ra thu được cho kết quả cuối cùng có kích thước 2$\times$2 (màu cam). Vùng màu xám trong hình chính là receptive field của một đặc trưng trên kết quả cuối cùng này.
% 	\vfill
% 	\begin{figure}[h!]
% 		\centering
% 		\begin{tikzpicture}[thick, scale=.95]
	\definecolor{blue}{RGB}{48, 137, 201};
\definecolor{cyan}{RGB}{43, 161, 152};
\definecolor{orange}{RGB}{249, 176, 53};

% \drawtile#x#y#size#color#opacity
\def\drawtile#1#2#3#4#5{
	\filldraw[fill=#4, opacity=#5] 
		(#1, #2) -- (#1 + #3, #2) -- (#1 + 1.5 * #3, #2 + #3 / 2) -- (#1 + #3 / 2, #2 + #3 / 2) -- cycle;
}

% \drawgrid#x#y#size#style
\def\drawgrid#1#2#3#4{
	\foreach \i in {0, ..., #3}{
		\draw[#4] (#1 + \i, #2) -- (#1 + \i + #3 / 2, #2 + #3 / 2); % vertical line
		\draw[#4] (#1 + \i / 2, #2 + \i / 2) -- (#1 + \i / 2 + #3, #2 + \i / 2); % horizontal line
	}
}

% \drawline#x1#y1#x2#y2#color
\def\drawline#1#2#3#4#5{
	\draw[#5] (#1, #2) -- (#3, #4);
}
	
	% layer 1
	\drawgrid{0}{0}{11}{dashed}
	\drawtile{4.5}{1.5}{5}{blue}{1.0}
	\drawgrid{4.5}{1.5}{5}{}
	\drawtile{0}{0}{7}{black}{0.2}
	\drawtile{3}{1}{3}{black}{0.3}
	
	% background connection
	\drawline{3}{1}{6}{6.5}{cyan}
	\drawline{6}{1}{7}{6.5}{cyan}
	\drawline{4.5}{2.5}{6.5}{7}{cyan}
	\drawline{7.5}{2.5}{7.5}{7}{cyan}
	\drawline{0}{0}{6.75}{9}{orange}
	\drawline{3.5}{3.5}{7.25}{9.5}{orange}
	\drawline{10.5}{3.5}{8.25}{9.5}{orange}
	
	% layer 2
	\drawgrid{4.5}{6}{5}{dashed}
	\drawtile{6}{6.5}{3}{cyan}{1.0}
	\drawgrid{6}{6.5}{3}{}
	\drawtile{4.5}{6}{3}{black}{0.2}
	
	% foreground connnection
	\drawline{9}{7.5}{8.25}{9.5}{orange}
	\drawline{7}{0}{7.75}{9}{orange}
	
	% layer 3
	\drawtile{6.75}{9}{2}{orange}{1.0}
	\drawgrid{6.75}{9}{2}{}
\end{tikzpicture}
% 		\caption[Receptive field của phép toán Convolution có kích thước filter $3\times 3$, stride 2 và padding ``same''.]{Receptive field của phép toán convolution\index{Convolution} có kích thước filter\index{Filter} $3\times 3$, stride\index{Stride} 2  và padding\index{Padding} ``same''\sourcefig{\cite{le2017receptive}}.}
% 		\label{fig:receptive_field}
% 	\end{figure}

% \subsection{Lớp Batchnorm}
% \label{subsec:lop_batchnorm}
% 	Thông thường chúng ta chuẩn hoá dữ liệu đầu vào bằng cách điều chỉnh và co giãn miền giá trị của chúng để việc học không bị thiên vị về một thuộc tính bất kỳ nào của dữ liệu. Ví dụ, nếu dữ liệu của chúng ta có một thuộc tính với miền giá trị trong khoảng từ 0 đến 1\linebreak và một thuộc tính khác có miền giá trị trong khoảng từ 0 đến 1000 thì chúng ta nên thực hiện bước chuẩn hoá dữ liệu để việc huấn luyện đạt kết quả tốt hơn. Tuy nhiên việc chuẩn hoá này chỉ mới được áp dụng trên dữ liệu đầu vào. Ioffe và Szegedy \cite{ioffe2015batch} đã đề xuất việc chuẩn hoá dữ liệu nên được thực hiện trên cả những lớp ẩn của kiến trúc mạng.
	
% 	Dữ liệu sẽ được chuẩn hoá nhờ lớp batchnorm\index{Batchnorm} trước khi đi qua một lớp mạng, giúp việc học trên từng lớp mạng có tính độc lập cao hơn, ít bị phụ thuộc bởi giá trị đầu ra của lớp mạng phía trước như trước đây. Đồng thời, việc sử dụng batchnorm cho phép chúng ta sử dụng giá trị learning rate\index{Learning rate} (tốc độ học) lớn hơn vì lớp batchnorm đảm bảo dữ liệu đầu vào của một lớp mạng không quá cao hoặc quá thấp.
	
% 	Lớp batchnorm thực hiện chuẩn hoá dữ liệu đầu ra của một lớp mạng bằng cách lấy giá trị đầu ra trừ đi giá trị trung bình (mean\index{Mean}) rồi chia cho độ lệch chuẩn (standard deviation) của chính nó. Chi tiết về giải thuật batchnorm được mô tả trong \autoref{fig:batchnorm}.
% 	\begin{figure}[h!]
% 		\centering
% 		\fbox{%
% 			\parbox{.6\textwidth}{%
% 				\vspace{-4mm}
% 				\begin{tabbing}
% 					\hspace{2cm}\=\kill
% 					\textbf{Input:} \>Values of $x$ over a mini-batch: $\mathcal{B}=\{x_{1...m}\}$;\\
% 					\>Parameters to be learned: $\gamma, \beta$\\
% 					\textbf{Output:} \>$\{y_i=\text{BN}_{\gamma,\beta}(x_i)\}$
% 				\end{tabbing}
% 				\begin{tabbing}
% 					\hspace{5mm}\=\hspace{6mm}\=\hspace{5mm}\=\hspace{4cm}\=\kill
% 					\>$\mu_\mathcal{B}$\>$\leftarrow$\>$\dfrac{1}{m}\sum\limits_{i=1}^{m}x_i$\>//mini-batch mean\\[3mm]
% 					\>$\sigma_\mathcal{B}^2$\>$\leftarrow$\> $\dfrac{1}{m}\sum\limits_{i=1}^{m}(x_i-\mu_\mathcal{B})^2$\>//mini-batch variance\\[3mm]
% 					\>$\hat{x}_i$\>$\leftarrow$\>$\dfrac{x_i-\mu_\mathcal{B}}{\sqrt{\sigma_\mathcal{B}^2+\epsilon}}$\>// normalize\\[3mm]
% 					\>$y_i$\>$\leftarrow$\>$\gamma\hat{x}_i+\beta\equiv\text{BN}_{\gamma, \beta}(x_i)$\>// scale and shift
% 				\end{tabbing}
% 			}%
% 		}
% 		\caption[Giải thuật Batchnorm áp dụng lên dữ liệu đầu vào của một mini-batch.]{Giải thuật Batchnorm áp dụng lên dữ liệu đầu vào của một mini-batch trong quá trình huấn luyện \sourcefig{\cite{ioffe2015batch}}.}
% 		\label{fig:batchnorm}
% 	\end{figure}

% \newpage
% \subsection{Lớp Sigmoid}
% \label{subsec:lop_sigmoid}
% 	Trong toán học, hàm sigmoid\index{Sigmoid} với phương trình toán học như sau
% 	\begin{equation}
% 	\sigma(x)=\dfrac{1}{1+e^{-x}}
% 	\label{eqn:sigmoid}
% 	\end{equation}
% 	là một hàm số biến một giá trị thực bất kỳ thành một giá trị thực trong phạm vi $(0, 1)$. Một số dương càng lớn sẽ cho ra giá trị gần với 1 và một số âm càng nhỏ sẽ cho ra giá trị gần với 0. Hàm sigmoid có thể được sử dụng sau một lớp convolution\index{Convolution} hoặc là lớp cuối cùng của kiến trúc mạng nhằm mục đích phân loại. Đồ thị biểu diễn hàm sigmoid được mô tả như \autoref{fig:sigmoid}.
% 	\begin{figure}[h!]
% 		\centering
% 		\input{figures/sigmoid}
% 		\caption{Đồ thị biểu diễn hàm sigmoid.}
% 		\label{fig:sigmoid}
% 	\end{figure}
% 	\vspace{-6mm}

% \subsection{Lớp ReLU}
% \label{subsec:lop_relu}
% 	ReLU (Rectified Linear Unit)\index{ReLU} do Mair và Hinton \cite{nair2010rectified} đề xuất với công thức toán học\linebreak như sau 
% 	\begin{equation}
% 		ReLU(x)=max(0, x),
% 		\label{eqn:relu}
% 	\end{equation}
% 	nghĩa là hàm ReLU đặt một ngưỡng tại 0 chỉ cho những giá trị dương đi qua. So với hàm Sigmoid\index{Sigmoid}, hàm ReLU giúp tăng tốc độ hội tụ khi huấn luyện bằng SGD (Stochastic Gradient Descent)\index{SGD}\nomenclature{SGD}{Stochastic Gradient Descent}. Một ưu điểm nữa của hàm ReLU là trong quá trình lập trình, hàm ReLU có thể hiện thực dễ dàng bằng cách áp một mặt nạ với ngưỡng bằng 0 cho cả quá trình lan truyền xuôi và lan truyền ngược. Tuy nhiên, chính việc áp mặt nạ làm cho gradient bằng $0$ tại một số điểm trong mạng dẫn đến gradient\index{Gradient} từ điểm đó về trước trong quá trình lan truyền ngược bằng $0$ và mạng không học được. Đồ thị biểu diễn hàm ReLU như \autoref{fig:relu}.
% 	\begin{figure}[h!]
% 		\centering
% 		\begin{tikzpicture}[domain=-4:4, scale=.9]
	% grid
	\draw[very thin, color=gray] (0, 0) grid (3, 3);
	
	% axes
	\draw[->,thick] (-4, 0) -- (4, 0) node[right] {$x$};
	\draw[->,thick] (0, 0) -- (0, 4) node[above] {$y$};
	\foreach \x in {-3, ..., 3}
		\node[anchor=north] at (\x, 0) {\x};
	\foreach \y in {1, ..., 3}
		\node[anchor=east] at (0, \y) {\y};
		
	% relu function
	\draw[very thick, color=red] plot (\x, { max(0, \x) }) 
		node[anchor=north,  xshift=-1cm] {$y=x$}
		node[anchor=south west, xshift=-8cm, yshift=-4cm] {$y=0$};
\end{tikzpicture}
% 		\caption{Đồ thị biểu diễn hàm ReLU.}
% 		\label{fig:relu}
% 	\end{figure}

% \subsection{Lớp Softmax}
% \label{subsec:lop_softmax}
% 	Trong toán học, hàm softmax\index{Softmax} (hay hàm trung bình mũ) là một hàm số biến không gian $K$ chiều với giá trị thực bất kỳ đến không gian $K$ chiều mang giá trị trong phạm vi $(0, 1)$. Phương trình toán học của hàm softmax được biểu diễn như sau
% 	\begin{equation}
% 	a_i=\dfrac{e^{z_i}}{\sum_{j=1}^{K}e^{z_j}},
% 	\label{eqn:softmax}
% 	\end{equation}
% 	trong đó, $z_i$ là giá trị đầu vào, $K$ là số khả năng khác nhau có thể xảy ra và $a_i$ là giá trị đầu ra của hàm softmax. Hàm softmax là một hàm đồng biến đảm bảo nếu giá trị $z_i$ càng lớn thì xác suất rơi vào khả năng thứ $i$ trong $K$ khả năng càng cao. Đồng thời, hàm này đảm bảo các giá trị đầu ra $a_i$ dương và tổng của chúng bằng $1$. \autoref{fig:softmax} mô tả một số giá trị đầu vào và đầu ra tương ứng của hàm softmax.
% 	\begin{figure}[h!]
% 		\centering
% 		\begin{tikzpicture}[thick]
	\definecolor{incolor}{RGB}{180, 181, 255}
	\definecolor{outcolor}{RGB}{255, 181, 181}
	
	% \drawcol#x#y#value#label#color
	\def\drawcol#1#2#3#4#5{
		\filldraw[fill=#5] (#1, {min(#2, #2 + #3)}) rectangle (#1 + .5, {max(#2, #2 + #3)}) 
			node[above, rotate=90, xshift=6mm] {#3};
		\node[anchor=north] at (#1 + .25, #2) {#4};
	}

	% \drawbar#x#y#value1#value2#value3#label#color
	\def\drawbar#1#2#3#4#5#6#7{
		\draw (#1, #2) -- (#1 + 3.5, #2);
		\drawcol{#1 + 0.5}{#2}{#3}{$#6_1$}{#7}
		\drawcol{#1 + 1.5}{#2}{#4}{$#6_2$}{#7}
		\drawcol{#1 + 2.5}{#2}{#5}{$#6_3$}{#7}
	}
	
	% input
	\drawbar{0}{5} { 2.0}{ 2.0}{ 2.0} {z} {incolor}
	\drawbar{4}{5} { 2.0}{ 1.5}{ 0.0} {z} {incolor}
	\drawbar{8}{5} {-1.0}{-1.5}{-2.0} {z} {incolor}
	
	% down arrow
	\draw[->] (1.75, 3) node[above] {$Softmax$} -- (1.75, 2) ;
	\draw[->] (5.75, 3) -- (5.75, 2);
	\draw[->] (9.75, 3) -- (9.75, 2);
	
	% output
	\drawbar{0}{0} {0.333}{0.333}{0.333} {a} {outcolor}
	\drawbar{4}{0} {0.574}{0.348}{0.078} {a} {outcolor}
	\drawbar{8}{0} {0.507}{0.307}{0.188} {a} {outcolor}
\end{tikzpicture}
% 		\caption[Một số ví dụ về đầu vào và đầu ra của hàm Softmax.]{Một số ví dụ về đầu vào và đầu ra của hàm Softmax \sourcefig{\cite{machinelearningcoban}}.}
		
% 		\label{fig:softmax}
% 	\end{figure}
	
% 	Trong lý thuyết xác suất, giá trị xuất ra của hàm softmax có thể được sử dụng để đại diện cho một loại phân phối -- đó là phân phối xác suất trên $K$ khả năng khác nhau có thể xảy ra.
	
% 	Hàm softmax\index{Softmax} được sử dụng trong nhiều phương pháp phân loại đa lớp như hồi quy logistic đa biến, biệt thức tuyến tính phân tích nhiều lớp và mạng nơ-ron. Đặc biệt, trong mạng nơ-ron, lớp softmax được sử dụng làm lớp cuối cùng của kiến trúc mạng để xác định độ chắc chắc trong kết quả dự đoán.
	
\newpage
\section{Machine Learning Model} 
\label{sec:machine_model}
	
\subsection{Logistic Regression}
\label{subsec:logistic_regression}
\subsubsection{Logistic Regression Model}
Predictive output of Linear Regression:
\begin{align}
	f(x) = w^T x
\end{align}

\subsubsection{Sigmoid Function}
Sigmoid function:
\begin{align}
    f(s) = \frac{1}{1 + e^{-s}} \triangleq \sigma(s)
\end{align}

\subsubsection{Optimize loss function}
The Stochastic Gradient Descent (SGD) algorithm will be used here.

\begin{align}
    & \frac{\partial z}{z(1-z)} = \partial s \\ 
	\Leftrightarrow & (\frac{1}{z} + \frac{1}{1 - z})\partial z = \partial s \notag \\  
	\Leftrightarrow & \log(z) - \log(1 - z) = s \notag \\ 
	\Leftrightarrow & \log \frac{z}{1 - z} = s \notag \\
	\Leftrightarrow & \frac{z}{1 - z} = e^s \notag \\
	\Leftrightarrow & z = e^s (1 - z) \notag \\
	\Leftrightarrow & z = \frac{e^s}{1 +e^s} = \frac{1}{1 + e^{-s}} = \sigma(s) \notag 
\end{align}

\subsubsection{Updated math formula for logistic sigmoid regression}
\begin{align}
    \frac{\partial J(\mathbf{w}; \mathbf{x}_i, y_i)}{\partial \mathbf{w}} = (z_i - y_i)\mathbf{x}_i
\end{align}
    
The updated formula (according to the SGD algorithm) for logistic regression:

\begin{align}
    \mathbf{w} = \mathbf{w} + \eta(y_i - z_i)\mathbf{x}_i
\end{align}
% \label{subsec:phep_toan_skeleton}
% 	Phép toán trích xuất khung xương đối tượng skeleton\index{Skeleton} là một trong các phép toán biến đổi hình thái được giới thiệu bởi Fisher và cộng sự \cite{hipr2thining}. Phép toán này được sử dụng để rút trích thành phần chính đại diện cho hình dạng của đối tượng trong ảnh nhị phân. Được ứng dụng trong nhận dạng mẫu (nhận dạng kí tự), nén ảnh, phát hiện lỗi trên sản phẩm công nghiệp (đứt đoạn).
	
% 	Gọi $\{nB\}, n= 0, 1, ...$ là một họ các hình khối với $B$ là một thành phần cấu trúc, ta có
% 	\begin{equation}
% 		nB = \underbrace{B\oplus \cdots \oplus B}_{n\ \text{lần}}.
% 		\label{eqn:skeleton_nb}
% 	\end{equation}
% 	Khi $n=0$, $nB=\{o\}$ với $o$ biểu diễn cho thành phần cấu trúc ban đầu. Phép toán skeleton lên ảnh nhị phân $A$ được định nghĩa như sau
% 	\begin{equation}
% 		S(A) = \bigcup_{n = 0}^N (A \ominus nB) - ((A \ominus nB) \ominus B)\oplus B,
% 		\label{eqn:skeleton}
% 	\end{equation}
% 	trong đó, $N$ là số lần lặp lớn nhất trước khi $A$ trở thành tập rỗng. $N$ được xác định theo công thức
% 	\begin{equation}
% 		N = \text{max}\{k|(A\ominus kB)\neq\varnothing\}.
% 	\label{eqn:skeleton_n}
% 	\end{equation}
	
% 	\autoref{fig:skeleton2d_apply} là một ví dụ áp dụng phép toán skeleton để rút trích khung xương đối tượng trong ảnh nhị phân.
% 	\begin{figure}[h!]
% 		\hfill
% 		\begin{subfigure}[b]{0.475\textwidth}
% 			\centering
% 			\input{figures/morphology_skeleton2d_before}
% 			\caption{}
% 			\label{fig:skeleton2d_apply_before}
% 		\end{subfigure}
% 		\begin{subfigure}[b]{0.475\textwidth}
% 			\centering
% 			\begin{tikzpicture}[thick, scale=.45]
	\def\colors{{
		"000 000 000", % black
		"255 255 255"  % white
	}}

	% \drawsquare#x#y#label#textcolor#fillcolor
	\def\drawsquare#1#2#3#4#5{
		\filldraw[fill=#5] (#1, #2) rectangle (#1 + 1, #2 + 1) node[pos=.5] {\textcolor{#4}{\scriptsize #3}};
	}
	
	% values
	\def\values{{
		0, 0, 0, 0, 0, 0, 0, 0, 0, 0, 0, 0, 0, 0, 0, 0,
		0, 0, 0, 0, 0, 0, 0, 0, 0, 0, 0, 0, 0, 0, 0, 0,
		0, 1, 1, 0, 0, 0, 0, 0, 0, 0, 0, 0, 0, 0, 0, 0,
		0, 0, 0, 1, 0, 0, 0, 0, 0, 0, 0, 0, 0, 0, 0, 0,
		0, 0, 0, 1, 0, 0, 0, 0, 0, 0, 0, 0, 0, 0, 0, 0,
		0, 0, 0, 1, 0, 0, 0, 0, 0, 0, 0, 0, 0, 0, 0, 0,
		0, 0, 0, 1, 0, 0, 0, 1, 1, 0, 0, 0, 0, 0, 0, 0,
		0, 0, 0, 1, 1, 1, 1, 0, 0, 1, 0, 0, 0, 0, 0, 0,
		0, 0, 0, 1, 0, 0, 0, 0, 0, 0, 1, 0, 0, 0, 0, 0,
		0, 0, 0, 1, 0, 0, 0, 0, 0, 0, 0, 1, 0, 0, 0, 0,
		0, 0, 0, 1, 0, 0, 0, 0, 0, 0, 0, 0, 1, 0, 0, 0,
		0, 0, 1, 0, 0, 0, 0, 0, 0, 0, 0, 0, 1, 0, 0, 0,
		0, 0, 1, 0, 0, 0, 0, 0, 0, 0, 0, 0, 1, 1, 0, 0,
		0, 0, 1, 0, 0, 0, 0, 0, 0, 0, 0, 0, 0, 1, 0, 0,
		0, 0, 0, 1, 1, 1, 0, 0, 0, 0, 0, 0, 0, 0, 0, 0,
		0, 0, 0, 0, 0, 0, 0, 0, 0, 0, 0, 0, 0, 0, 0, 0,
	}}
	
	% pixel
	\foreach \y in {0, ..., 15} {
		\foreach \x in {0, ..., 15} {
			\def\value{\values[16 * \y + \x]}
			\pgfmathparse{\colors[\value]};
			\definecolor{fillcolor}{RGB}{\pgfmathresult};
			\pgfmathparse{\colors[1 - \value]};
			\definecolor{textcolor}{RGB}{\pgfmathresult};
			\drawsquare{\x}{15 - \y}{\pgfmathparse{\value}\pgfmathresult}{textcolor}{fillcolor}
		}
	}
\end{tikzpicture}
% 			\caption{}
% 			\label{fig:skeleton2d_apply_after}
% 		\end{subfigure}
% 		\hfill\null
% 		\caption[Áp dụng phép toán Skeleton để rút trích khung xương đối tượng trong không gian 2D.]{Áp dụng phép toán Skeleton để rút trích khung xương đối tượng trong không gian 2D. \subref{fig:skeleton2d_apply_before} ảnh nhị phân ban đầu. \subref{fig:skeleton2d_apply_after} ảnh nhị phân sau khi trích xuất khung xương đối tượng. \sourcefig{\cite{hipr2thining}}.}
% 		\label{fig:skeleton2d_apply}
% 	\end{figure}
	
% 	Trong luận văn này chúng tôi sử dụng kỹ thuật rút trích khung xương đối tượng trong không gian 3D được đề xuất bởi Lee và cộng sự \cite{lee1994building} để tìm đường chính giữa của mạch máu trong \autoref{sec:tim_duong_chinh_giua_va_diem_phan_nhanh}. \autoref{fig:skeleton3d_apply} minh hoạ việc áp dụng phép toán skeleton làm mảnh đối tượng trong không gian 3D.
% 	\begin{figure}[h!]
% 		\centering
% 		\begin{subfigure}[b]{0.49\textwidth}
% 			\centering
% 			\includegraphics[width=.8\textwidth]{figures/morphology_skeleton3d_before}
% 			\caption{}
% 			\label{fig:skeleton3d_apply_before}
% 		\end{subfigure}
% 		\hfill
% 		\begin{subfigure}[b]{0.49\textwidth}
% 			\centering
% 			\includegraphics[width=.6\textwidth]{figures/morphology_skeleton3d_after}
% 			\caption{}
% 			\label{fig:skeleton3d_apply_after}
% 		\end{subfigure}%
% 		\caption[Áp dụng phép toán Skeleton để rút trích khung xương đối tượng trong không gian 3D.]{Áp dụng phép toán Skeleton để rút trích khung xương đối tượng trong không gian 3D. \subref{fig:skeleton3d_apply_before} đối tượng ban đầu. \subref{fig:skeleton3d_apply_after} khung xương được tạo ra từ phép toán skeleton. \sourcefig{\cite{lee1994building}}.}
% 		\label{fig:skeleton3d_apply}
% 	\end{figure}

\section{Deep Neural Network}
\label{sec:deep_neural_network}
With their excellent results, broad applicability, and vast growth potential, neural networks are currently the most advanced advancement in artificial intelligence.
Feedforward Neural Networks (FNN) and Convolution Neural Networks (CNN) are widely used to make predictions with independent data input. In CNN (Figure 2.3), each input image is passed through a series of convolutional layers (Filters, Pooling, and Fully-connected layers) to extract features before being classified using the Softmax function\footnote{Softmax function:$ \sigma (\overrightarrow{z})_i \: = \: (e^{z_i})/ \sum_{j=1}^{K} e^z_i  $}.
\begin{figure}[h!]
	\centering
	\includegraphics[width=\linewidth, height=10cm,keepaspectratio]{figures/CNN.png}
	\caption{CNN Architecture}
\end{figure}	

\emph{Layers} are the building components of deep neural networks such as CNN and others. A \emph{layer} is a broad phrase that refers to a group of 'nodes' that function together at a given level within a neural network. \emph{Layers} are classified into three types: \emph{input layer},\emph{ hidden layers}, and \emph{output layers}.

The \emph{input layer} contains raw data (each variable as a ‘node’) \\
In neural networks, black magic happens in the \emph{hidden layer}(s). By minimizing an error/cost function, each layer attempts to learn different elements of the data. The context of 'image recognition', such as a face, is the most intuitive way to understand these levels. The first layer may learn edge detection, the second eye, the third nose, and so on. This isn't exactly what's going on, but the idea is to split the problem down into components that different levels of abstraction can piece together, much to how our own brains work (hence the name neural networks).

The \emph{output layer} is the simplest, usually consisting of a single output for classification problems. Even though it is a single 'node,' it is nevertheless regarded as a layer in a neural network because it might contain numerous nodes. We use six types of layers given by the TensorFlow framework in this thesis: \textbf{Embedding} layer, \textbf{Conv1D} layer, \textbf{MaxPooling1D} layer, \textbf{Dropout} layer, \textbf{Flatten} layer, and \textbf{Dense} layer.
\newpage
\subsection{Gradient descent}
\label{subsec:gradient_descent}
\emph{Gradient descent} is an optimization algorithm that is used to minimize a function by iteratively traveling in the direction of the steepest descent as defined by the gradient's negative. Gradient descent is used in machine learning to update the parameters of our model, especially weights in neural networks.
\begin{figure}[!h]
	\centering
	\includegraphics[width=\linewidth, height=10cm,keepaspectratio]{figures/gradient descent.png}
   \caption{Gradient descent demonstration}
\end{figure}
\subsection{Learning Rate}
\label{subsec:learning_rate}
The learning rate is the size of each step in each gradient descent cycle. We can cover more territory per step with a high learning rate, but we risk overshooting the lowest spot because the slope of the hill is continually changing. We may reliably go in the direction of the negative gradient with a very low learning rate because we are recalculating it so frequently. A low learning rate is more exact, but calculating the gradient takes time, so we will take a long time to reach the bottom. An example of the learning rate is shown in Figure 2.5.
\newpage
\begin{figure}[!h]
	\centering
	\includegraphics[width=\linewidth, height=10cm,keepaspectratio]{figures/learning rate.png}
   \caption{Learning rate}
\end{figure}

\subsection{Loss Function}
\label{subsec:loss_function}
A Loss Function (or cost function) indicates how well the model predicts a given set of parameters. Index of the cost function The loss function has its curve and gradients. The slope of this curve indicates how we should adjust our parameters to improve the model's accuracy. If the cost ever rises, we must reduce the value of the learning rate; if the cost falls slowly, we must increase the value of the learning rate. Figure 2.6 shows examples of loss function behavior based on the learning rate.
\begin{figure}[!h]
	\centering
	\includegraphics[width=\linewidth, height=10cm,keepaspectratio]{figures/loss function DL.png}
   \caption{Loss function behaviour with different learning rate}
\end{figure}

\subsection{Gradient descent optimizer}
\label{gradient_optimizer}
A method that computes adaptive learning rates for each parameter is Adaptive Moment Estimation (Adam). Adam, like Adadelta and RMSprop, preserves an exponentially decaying average of past squared gradients $v_t$ in addition to an exponentially decaying average of past gradients $m_t$. Whereas momentum can be thought of as a ball rolling down a hill, Adam behaves more like a heavy ball with friction, preferring flat minima on the error surface. The decaying averages of past and past squared gradients, $m_t$ and $v_t$, are computed as follows:
\begin{align}
    m_t \: = \: \beta_1 m_{t-1} \: + \: (1-\beta_1)g_{t}\\
    v_t \: = \: \beta_2 v_{t-1} \: + \: (1-\beta_2)g^2_t
\end{align}


$m_t$ and $v_t$ are estimates of the first moment (the mean) and the second moment (the uncentered variance) of the gradients respectively, hence the name of the method. As $m_t$ and $v_t$ are initialized as vectors of 0's, the authors of Adam observe that they are biased towards zero, especially during the initial time steps, especially when the decay rates are small.
\begin{align}
    \hat{m}_t \: = \: \frac{m_t}{1\:-\:\beta^t_1} \\
 \hat{v}_t \: = \: \frac{v_t}{1\: - \: \beta^t_2}
\end{align}



They then use these to update the parameters:
\begin{align}
\theta_{t+1} \: = \: \theta_t \: - \: \frac{\eta}{\sqrt{\hat{v}_t} \: +\: \epsilon} \hat{m}_t 
\end{align}
% \subsection{Thư viện VTK}
% \label{subsec:thu_vien_vtk}
% 	\nomenclature{VTK}{The Visualization Toolkit}\index{VTK}VTK (The Visualization Toolkit)\footnote{\url{https://pypi.org/project/vtk/}}là một thư viện mã nguồn mở cung cấp cho các nhà phát triển một bộ công cụ đa dạng phục vụ cho đồ hoạ máy tính trong không gian ba chiều, xử lý hình ảnh và trực quan hoá. Thư viện này được hiện thực từ nhiều ngôn ngữ khác nhau như C++, Java và Python. Nó hỗ trợ nhiều giải thuật trực quan hoá bao gồm phương pháp vô hướng, phương pháp vec-tơ, phương pháp kết cấu và thể tích. Đặc biệt, thư viện này còn hỗ trợ các kỹ thuật mô hình hoá nâng cao như mô hình ẩn, thu giảm đa giác, làm mịn bề mặt, tạo đường viền và tam giác phân Delaunay.
	
% 	Trong luận văn này, chúng tôi sử dụng thư viện VTK trên ngôn ngữ Python nhằm trích xuất và làm mịn lưới bề mặt hệ thống mạch máu cũng như đường chính giữa mạch máu và các điểm phân nhánh từ kết quả dự đoán, phục vụ quá trình trực quan hoá sẽ được đề cập tới trong \autoref{sec:truc_quan_hoa_ket_qua_thi_nghiem}.

% \subsection{Phần mềm Slicer}
% \label{subsec:pham_mem_Slicer}
% 	\index{Slicer}Slicer\footnote{\url{https://www.slicer.org/}}là ứng dụng mã nguồn mở phục vụ cho tin học về hình ảnh y tế, xử lý hình ảnh y khoa và trực quan hoá trong không gian ba chiều. Được xây dựng trong hơn hai thập kỷ qua với sự hỗ trợ của các viện y tế quốc gia (National Institutes of Health) và cộng đồng các nhà phát triển trên toàn thế giới, trong đó phải kể đến sự đóng góp của cộng đồng quốc tế các nhà khoa học, bao gồm cả kỹ thuật và y sinh. Slicer mang đến các công cụ xử lý đa nền tảng miễn phí, mạnh mẽ cho các bác sĩ, các nhà nghiên cứu và cộng đồng.
	
% 	Trong luận văn này, chúng tôi sử dụng ứng dụng Slicer để trực quan hoá kết quả thu được từ hệ thống, nhằm đưa ra những nhận xét mang tính định tính chất lượng, giúp kết quả đánh giá được toàn diện hơn. Chúng tôi sẽ đề cập rõ hơn trong \autoref{sec:truc_quan_hoa_ket_qua_thi_nghiem}.